\documentclass[11pt,a4paper]{article}
\usepackage[T1]{fontenc}
\usepackage[utf8]{inputenc}
\usepackage{amsmath}
\usepackage{siunitx}

\begin{document}

\section*{Exercice 4 -- Q10}
\textbf{Q10 :} Quelle puissance en mW consomme chaque processeur à la fréquence maximale ?

\subsection*{Données}
\begin{itemize}
  \item Cortex A7 : \SI{0.10}{mW/MHz}, $f_{\max}=\SI{1.0}{GHz}=\SI{1000}{MHz}$.
  \item Cortex A15 : \SI{0.20}{mW/MHz}, $f_{\max}=\SI{2.5}{GHz}=\SI{2500}{MHz}$.
\end{itemize}

\subsection*{Calcul}
La puissance à fréquence maximale est donnée par :
\[
P = \left(\frac{\mathrm{mW}}{\mathrm{MHz}}\right)\times f\,(\mathrm{MHz}).
\]

\paragraph{Cortex A7}
\[
P_{A7} = 0.10\ \frac{\mathrm{mW}}{\mathrm{MHz}} \times 1000\ \mathrm{MHz} = \SI{100}{mW}.
\]

\paragraph{Cortex A15}
\[
P_{A15} = 0.20\ \frac{\mathrm{mW}}{\mathrm{MHz}} \times 2500\ \mathrm{MHz} = \SI{500}{mW}.
\]

\subsection*{Réponse}
\begin{itemize}
  \item Cortex A7 : \SI{100}{mW} à $\SI{1.0}{GHz}$.
  \item Cortex A15 : \SI{500}{mW} à $\SI{2.5}{GHz}$.
\end{itemize}

\end{document}
