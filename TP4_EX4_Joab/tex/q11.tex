\documentclass[11pt,a4paper]{article}

\usepackage[utf8]{inputenc}
\usepackage[T1]{fontenc}
\usepackage{lmodern}
\usepackage{geometry}
\usepackage{graphicx}
\usepackage{booktabs}
\usepackage{siunitx}

\geometry{margin=2.2cm}
\graphicspath{{figs/q11/}}
\sisetup{locale=FR}

\title{TP4 -- Exercice 4 (Q11)\\\small Efficacit\'e \`energ\'etique en fonction de la taille des caches L1}
\author{\small Nom: \underline{\hspace{5cm}} \qquad Groupe: \underline{\hspace{2cm}}}
\date{\small Ann\'ee 2026}

\begin{document}
\maketitle

\section{Objectif}
La question Q11 demande de calculer, pour chaque configuration de cache L1, l'\emph{efficacit\'e \`energ\'etique} des deux processeurs (Cortex A7 et Cortex A15) \`a leur fr\'equence maximale.

\section{M\'ethode}
D'apr\`es l'\'enonc\'e, la consommation est de \SI{0.10}{mW/MHz} pour le Cortex A7 (\SI{1.0}{GHz}) et de \SI{0.20}{mW/MHz} pour le Cortex A15 (\SI{2.5}{GHz}).
On en d\'eduit les puissances \`a $f_{\max}$ (r\'esultat Q10) :
\begin{itemize}
  \item $P_{A7}=\SI{100}{mW}$
  \item $P_{A15}=\SI{500}{mW}$
\end{itemize}

L'efficacit\'e \`energ\'etique est d\'efinie par :
\[
\text{Efficacit\'e \`energ\'etique} = \frac{IPC}{P\,(mW)}.
\]
Les valeurs d'IPC proviennent des simulations gem5 (Q4 pour A7, Q5 pour A15) effectu\'ees \emph{jusqu'\`a la fin du programme}.

\section{R\'esultats}
\subsection{Synth\`ese}
\begin{center}
% Auto-generated by analysis/q11_energy.py
\begin{tabular}{lrrrr}\toprule
Application & Best L1 A7 (KB) & IPC/mW A7 & Best L1 A15 (KB) & IPC/mW A15\\\midrule
Dijkstra & 16 & 0.002834 & 32 & 0.002593\\
Blowfish (enc+dec) & 16 & 0.002992 & 32 & 0.003345\\
\bottomrule\end{tabular}

\end{center}

\subsection{Dijkstra}
\begin{figure}[h]
\centering
\includegraphics[width=0.80\linewidth]{q11_energy_dijkstra.png}
\caption{Q11 : efficacit\'e \`energ\'etique (IPC/mW) vs taille L1 (Dijkstra).}
\end{figure}

\subsection{Blowfish (enc+dec)}
\begin{figure}[h]
\centering
\includegraphics[width=0.80\linewidth]{q11_energy_blowfish_total.png}
\caption{Q11 : efficacit\'e \`energ\'etique (IPC/mW) vs taille L1 (Blowfish enc+dec).}
\end{figure}

\section{Analyse}
Comme $P$ est fix\'e par processeur \`a $f_{\max}$, l'efficacit\'e \`energ\'etique suit directement les variations d'IPC avec la taille de L1.
Les meilleurs points correspondent donc aux tailles de L1 qui maximisent l'IPC pour chaque application et chaque processeur.

\end{document}
